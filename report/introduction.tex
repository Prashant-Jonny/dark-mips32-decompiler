\section{Introduction}

In the MIPS32 architecture, all machine instructions are represented
as 32-bit numbers. This article presents a MIPS32-decompiler that,
when passed 32-bit numbers which either partially or completely
represent MIPS32 instructions yields a series of different
representations of the same instruction.

Specifically, MIPS32 instructions are to be read from a file with
numbers either in decimal or hexadecimal form. For each number in the
input file, the disassembler will produce the following:

\begin{itemize}
  \item The number from the input file.
  \item The format of the instruction (R, I, or J).
  \item The decomposed representation in decimal.
  \item The decomposed representation in hexadecimal.
  \item The representation of the instruction in mnemonic format,
    using register abbreviations wherever possible (e.g.,
    \texttt{\$t0} instead of \texttt{\$8}) and using decimal numbers
    whenever actual numbers are necessary.
\end{itemize}

In the following subsections an introduction of the terminology used
throughout this document is provided. Afterwards you may refer to this
section again if any of the above requirements seem foreign to you.

The rest of the document will be dedicated to a high-level description
of this solution accompanied with a guide on how to compile and run
the software.

The appendix contains an overview of all those instructions that the
decompiler is capable of comprehending.
