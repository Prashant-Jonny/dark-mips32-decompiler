In the previous section we established the terminology we will use
throughout the document. This terminology will also be used to
decompose our system that solves the problem at hand, viz. that
we will model our solution according to the problem domain.

Specifically, we will create a system such that any given
\tt{Instruction} knows, or can retrieve information
about,\footnote{Using its collaborators}

\begin{itemize}
\item Its numerical representation
\item Its own name
\item Its own format
\item Its decomposed representation(s)
\subitem Hexadecimal representation
\subitem Decimal representation
\item Its mnemonic representation
\end{itemize}

The \tt{Instruction} class is an abstraction providing an
encapsulation around the aforementioned points. Ultimately

Furthermore, the system will be divided into the self-explanatory
constituents

\begin{itemize}
\item \formatm: R, I, J
\item \inamem: All known instruction names, excluding
  pseudo-instructions.
\item \opcodem: An abstraction around the numerical opcode. Knows its
  associated format.
\item \decomposedm: Can decompose a 32-bit number into
  varying lengths.
\item \registerm: Associates a number with the register name.
\end{itemize}

The manner with which we determine the particular instruction, and
yield the correct output, becomes very closely related to our original
example for the \tt{mul} instruction. In the following example
we will parse another R-type instruction, namely \tt{sub}, this
time explaining the process more formally,

\subsection{Example: Decomposing \texttt{sub}}

Consider the 32-bit integer \tt{0x00012122}. In this example, we
will parse the number in a sequence of steps, before finally arriving
at that this number represents an instance of the \tt{sub}
instruction.

\subsubsection{Determining \texttt{Opcode} from a 32-bit integer}
First, we have that from the \tt{Opcode} may be determined
by the 6 left-most bits of the number, hence we would like to
have a function $f$ such that,

\begin{equation*}
        f: \textrm{32-bit integer} \to \tt{Opcode}
\end{equation*}

In Java we can represent this by having a class, \opcodem\footnote{
\tt{src/main/java/se.filipallberg.dark.mips32decompiler.instruction.opcode.Opcode}}
with the method,

\begin{minted}{java}
        public static Opcode fromInstruction(int instruction);
\end{minted}

The \opcodem abstraction will intermittently be represented in
numberical form, so in this case \mintinline{java}{Opcode.fromInstruction(0x00012122)}
will occassionally be written out as \tt{0x00}.

\subsubsection{Determining the \texttt{Format} from an \texttt{Opcode}}

As previously stated, for any defined \opcodem the
format is always known, for legibilitywe will adopt the following
notation to describe this function.\footnote{And that methods are
routinely renamed}

\begin{minted}{haskell}
getFormat :: Opcode -> Format
\end{minted}

which means that getting the format is a function \emph{from} \opcodem
to \formatm.

This also means that we will write out

\begin{minted}{java}
  public static Opcode fromInstruction(int instruction);
\end{minted}

as

\begin{minted}{haskell}
fromInstruction :: Integer32 -> Opcode
\end{minted}

When appropriate, we will disregard the names and just write

\begin{verbatim}
Integer32 -> Opcode
\end{verbatim}

Recall that so far we have applied the following steps,

\begin{itemize}
\item \tt{\integer -> \opcode}, which yields \tt{0x00}
\item \tt{\opcode -> \format}, which yields the R-format
\end{itemize}

Now while the above two steps are always applicable regardless of the instruction
there lies a cross-roads in the next step that we have to take into account.

\subsubsection{Getting the \texttt{InstructionName} from the \texttt{Format}}

Recall from earlier that the \opcode alone is not always a unique identifier of
the instruction. For I-format instructions it is, but for R-format instructions
it is not!

But, for any given \formatm we may always identify the name of the
an \instructionm, i.e. the \inamem as long as we can supply the format
with the numerical representation of the instruction.

In the case of the formats where the \opcodem uniquely identifies the
instruction the specific \formatm only confers with the \opcodem to
supply the \inamem. In those formats where the \opcodem is not
adequate to identify the instruction \formatm may confer with any bits
necessary to provide the \inamem.\footnote{For the opcodes \tt{op=0x00}, \tt{op=0x01},
and \tt{op=0x1c} it is necessary to consult a sencond field in order to identify
the instruction completely (\tt{funct} in the case of \tt{op=x00} or \tt{ox1c} and
\tt{rt} in the case of \tt{op=0x01})}.

Using our previous notation convention we will write,

\begin{minted}{haskell}
getName :: Format -> Integer32 -> InstructionName
\end{minted}

and the way we interpret is a left-associative function application, which
means that reading from the left we first apply \integer as an argument
to \formatm, i.e. \mij{Format(Integer32)} which returns an \inamem.

So, our initial two steps gave us the R-format, which when given a
32-bit integer (in this case \tt{0x00012122}) will give us
the \inamem: \mij{SUB}.

The \formatm also affords us the definition of how to decompose an
instruction of that format.

\begin{minted}{haskell}
getDecomposedRepresentation :: Format -> Integer32 -> DecomposedRepresentation
\end{minted}

\subsubsection{Review}

So far we have retrieved the \opcodem from \tt{0x00012122}, and from the
\opcodem we got the \formatm.

In turn, the \formatm gave us both the \decomposedm and the \inamem.

\subsection{Example: Decomposing \tt{sub} (continued)}

Before the above review we retrieved the \decomposedm. A \decomposedm
can represent itself in different bases, specifically in hex and decimal,
as a \mij{String}.

\begin{minted}{haskell}
hex :: DecomposedRepresentation -> String
decimal :: DecomposedRepresentation -> String
\end{minted}

where if we apply \mi{haskell}{hex} to the \decomposedm we have
we get \tt{[0 0 1 4 4 0x22]}.\footnote{Remember that an R-type instruction
decomposes into bitfields of length $(6, 5, 5, 5, 5, 6)$} and applying
\mintinline{haskell}{decimal} yields \tt{[0 0 1 4 4 34]}.

Recall from Fig. \ref{fig:mips32-decompiler} that for any 32-bit integer
representing a valid MIPS32-instruction the output we intend to produce is

\begin{itemize}
  \item The supplied 32-bit integer
  \item the format of the instruction
  \item the hexadecimal and decimal decomposed representation
  \item the mnemonic representation using register
          abbreviations wherever possible, and using decimal numbers
          when actual numbers are necessary.
\end{itemize}

We have docked off the first three items, and only getting the
mnemonic representation remains.\footnote{The first item is trivial}

\subsubsection{Mnemonic patterns}

We have not covered this before but the mnemonic representation of
instructions in the MIPS32 instruction set has a tendency to follow
a certain pattern, and these patterns are dependent upon the format of
the particular instruction.

For an example, the instructions

\begin{verbatim}
add, addu, sub, mul, and, or, nor, xor, movn, movz
\end{verbatim}

among possibly others, have the common pattern \emph{$\langle$iname
rd, rs, rt$\rangle$} with
\emph{iname} the instruction name.\footnote{\inamem}

\newcommand{\pattern}[1]{\emph{$\langle$#1$\rangle$}}

Other common patterns include 

\begin{itemize}
\item \pattern{iname rd, rs}
\item \pattern{iname rd, rt, rs}
\item \pattern{iname, rd, rt, shamt}
\end{itemize}

for R-format instructions and

\begin{itemize}
\item \pattern{iname, rt, rs, imm}
\item \pattern{iname rt, imm}
\item \pattern{iname, rt, rs(addr)}
\end{itemize}

for I-format instructions.

Each \inamem is associated with a particular \pattern! We will call
these patterns a \mij{MnemonicRule}, so we have that

\begin{minted}{haskell}
getMnemonicRule :: InstructionName -> MnemonicRule
\end{minted}

and that a \mij{MnemonicRule} when applied to a \decomposedm
yields a \mij{Mnemonic}.\footnote{This is just a type synonym for a \texttt{String}}

\begin{minted}{haskell}
getMnemonic :: InstructionName -> MnemonicRule -> DecomposedRepresentation -> Mnemonic
\end{minted}

\subsection{An decompilation algorithm}

Using the functions expressed in the previous sections we are now ready to
define an algorithm for our 32-bit integers.

\LetLtxMacro\i\textit
\newcommand{\get}[2]{\tt{#1} \i{#2} $\gets$ }
\newcommand{\of}[1]{(\i{#1})}
\newcommand{\from}[1]{\i{#1}}

\begin{algorithm}
\caption{Parsing/Decompilation algorithm}\label{algo:decompile}
\begin{algorithmic}[1]
\Procedure{Parse}{32-bit integer \i{int}}
\State \get{Opcode}{o}\from{getOpcode}\of{int}
\State \get{Format}{format}\from{getFormat}\of{o}
\State \get{InstructionName}{iname}\from{getName}\of{format, int}
\State \tt{DecomposedRepresentation} d
\State \get{}{d}\from{getDecomposedRepresentation}\of{format, int}
\State \get{String}{dec}\from{decimal}\of{d}
\State \get{String}{hex}\from{hex}\of{d}
\State \get{MnemonicRule}{r}\from{getMnemonicRule}\of{iname}
\State \get{Mnemonic}{mnemonic}\from{getMnemonic}\of{iname, r, d} \\
\Return (int, format, dec, hex, mnemonic)
\EndProcedure
\end{algorithmic}
\end{algorithm}
