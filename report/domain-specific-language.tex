\section{Domain-specific language}

In the previous section.~\ref{section:problem-statement} we described
the \emph{core} issues that our solution needs to address. 

Summarizingly, we found that we want to be able to

\begin{itemize}
\item Get the six left-most bits from a 32-bit integer,
  i.e. the numerical opcode.
\item Infer the Format of the instruction from the opcode.
\item When necessary consult additional bitfields in the 32-bit integer to
      determine the particular instruction.
\item Validate other bitfields when necessary.
\item Convert the numerical representation into a mnemonic representation.
\end{itemize}

The first two items are trivial, the first is the result of a
bit-shift operation and as alluded to previously the Format is always
inferrable from the Opcode, which means that we can either let each
opcode know its associated format or similarily associate formats with
sets of opcodes elsewhere. The specific approach will be covered
later, as will the reason it was chosen.

The subset of the MIPS32-instruction set that we have
implemented,\footnote{Floating point operations and a few others have
  been omitted} is made up of a large number of operations so we would
like to design our system in such a manner that all of this
\emph{knowledge} pertaining to a particular instruction, viz. how to
validate it and represent it, be stored in a single place.
